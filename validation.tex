%\documentclass[prd, nofootinbib, floatfix, 12pt]{revtex4}
%\documentclass[useAMS,usenatbib,11pt,preprint]{aastex}
\documentclass[]{article}

\usepackage[paperwidth=8.5in,paperheight=11in,centering,hmargin=1in,vmargin=1in]{geometry}
\usepackage[round]{natbib}
\usepackage{float}
\usepackage{amsmath}
\usepackage{amsbsy}

\topmargin0.0cm
\textheight8.5in

\input epsf
\usepackage{amsmath,amssymb}
\usepackage{graphicx}
\usepackage[margin=0in]{caption}
\usepackage{subfigure}
\usepackage{epsfig}
\usepackage{color}
%\usepackage{ulem}
%\usepackage{epstopdf}

\renewcommand{\topfraction}{0.95}
\renewcommand{\bottomfraction}{0.95}





%%%%%%%%%%%%%%%%%%%%%%%%%%%%%%%%%%%%%%%%%%%%%%%%%%%%%%%%%%%%
%%%%%%%%%%%%%%%%%%%%%%%%%%%%%%%%%%%%%%%%%%%%%%%%%%%%%%%%%%%%
%%%%%%%%%%%%%%%%%%%%%%%%%%%%%%%%%%%%%%%%%%%%%%%%%%%%%%%%%%%%

\begin{document} 
\sloppy
\title
{The LSST Universe model and its validation}

%\pagerange{\pageref{firstpage}--\pageref{lastpage}}

\label{firstpage}

% \date{\today}

\maketitle 

\abstract{The Large Synoptic Survey Telescope (LSST) will be a
  wide-field telescope and multi-epoch imaging survey.  The LSST has
  an 8.4m aperture (6.7m effective) and a 9.6 deg$^{2}$ field of view.
  The large etendue of the LSST enables the project to image the
  visible sky every 3 days. To evaluate the performance of the
  delivered LSST, to design the algorithms used in processing the LSST
  data, and to enable the development of the science analysis
  techniques the LSST has turned to high fidelity simulations.  These
  simulations include a realistic model of the universe (base
  catalogs), a framework for querying and formatting the model at
  specific instances in time (catalogs framework), and a fast raytrace
  code for simulating the raw images (PhoSim).  In this document we
  describe the base catalogs and catalogs framework and validate them
  for use by the LSST.  The requirements against which we will
  validate are described in the document ``Requirements for the LSST
  Simulation Framework''. 
%\citep[hereafter  Requirements][]{requirements}.
  
\section{Introduction \label{sec:intro}}

A new generation of astronomical survey telescopes, the Dark Energy
Survey (DES), the Panoramic Survey Telescope \& Rapid Response System
(Pan-STARRS), EUCLID, the Visible and Infrared Survey Telescope for
Astronomy (VISTA), and the Large Synoptic Survey Telescope (LSST) are
now, or soon will be, surveying the universe in unprecedented
detail. Repeated observations of the same part of the sky, with
hundreds to thousands of observations over a period of ten years, will
enable a detailed study of the temporal universe (ranging from
transient sources such as supernovae and optical bursters, to periodic
variables such as Cepheids and RR-Lyrae stars, to moving sources such
as asteroids and high proper motion stars). Combined, these
observations will provide some of the deepest, large-scale surveys of
the universe ever undertaken and provide the ability to measure the
nature of dark energy with figures of merit 10-100 times better than
current surveys \citep[DETF][]{albrecht06}.

The stringent requirements on the statistical power of these
telescopes means that we will soon no longer be limited by shot noise
(i.e.\ the number of sources within a sample) but by how well we can
understand systematic uncertainties within our data streams. These
systematic effects can arise from the design of the telescope
(e.g.\ ghosting of images or scatter light), from the response of the
atmosphere (e.g.\ the stability of the point-spread-function or the
variability in the transmissivity of the sky), from the strategy used
to survey the sky (e.g.\ inhomogeneous sampling of astronomical light
curves), or from limitations in our analysis algorithms (e.g.\ due to
the finite processing power available approximations may need to be
made when characterizing the properties of detected
sources). Understanding which of these issues will impact the science
from a given telescope is critical if we hope to maximize their
scientific returns.

Simulations of the data flow from survey telescopes can provide a
critical role in understanding the capabilities of an astronomical
system and in optimizing its scientific returns. By providing data
with the expected characteristics of a survey well in advance of first
light, algorithms and statistical techniques can be optimized and
scaled to the expected data volumes or new statistical approaches can
be developed to improve the data analysis. 
In the following sections we describe the data used to define the universe
we expect the LSST to observe and validate the properties of the model
universe against the requirements of the project \citep{requirements}.

\section{Source Catalogs and the Catalog Framework}

\subsection{Framework}


The design of a framework to simulate the data expected from the LSST
requires flexibility and scalability \citep{connolly10}.
%(to enable
%data generation runs that range from a single CCD image of a
%gravitational lens to images from thousands of full focal planes that
%trace the expected observing cadence of the survey). 
This is accomplished by dividing the simulation workload into three
separate components: a base component that stores a model of the
universe (including the distribution of galaxies from a cosmological
simulation, the distribution of stars from a Galactic Structure model,
and a model for the asteroid populations within our Solar System), a
system for querying the underlying model of the universe using
simulations of sequences of LSST observations, and a framework for the
generation of images via the ray-tracing of individual photons.


Figure~\ref{fig:flow} shows an example of the flow of information
through the LSST simulation framework. 
%Simulations of sequences of
%LSST observations enable catalogs of LSST sources to be
%generated. These catalogs can be analyzed for different science
%programs or passed to a photon based image generator that create input
%images for the data management analysis pipelines.  
The design enables the generation of a wide range of data products:
from all-sky catalogs used in modeling the LSST calibration pipeline,
to time domain data used to characterize variability as a function of
signal-to-noise and temporal sampling, to sequences of images of
gravitational lenses from which to measure cosmological time delays.
In this document we focus on the first of the components of the
framework; the LSST universe model or base catalog and the mechanism
for querying this model. We refer to this system as {\it catsim}
throughout this document.  The base catalog is stored as a SQL
database (using a Microsoft SQLServer). Data are accessible through a
Python interface that uses SQLalchemy ({\tt
  http://www.sqlalchemy.org}) to provide a database agnostic view of
the sources. For any LSST pointing, sources can be queried as a
function of position and time with the returned data accounting for
any change in brightness due to variability. For large scale runs, the
base catalog is queried using sequences of observations derived from
the Operations Simulator \citep{cook09} (see also: {\tt
  http://www.lsst.org/lsst/opsim}).  The Operation Simulator simulates
LSST pointings that meet the cadence and depth requirements of the
LSST science cases while accounting for historical weather patterns
for Cerro Pachon and the visibility of the LSST footprint on the
sky. Each simulated pointing provides a position and time of the
observation together with the appropriate sky conditions (e.g. seeing,
moon phase and angle, and sky brightness). Positions of sources are
propagated to the time of observation (including the proper motion
information for stars and orbits for Solar System sources). Magnitudes
and source counts are derived using the atmospheric and filter
response functions appropriate for the airmass of the observation and
after applying corrections for source variability. The resulting
catalogs (instance catalogs) can be formatted for use in a science
application (e.g. measuring the proper motions of high velocity stars)
or fed to the final component of the simulation framework, the image
simulator \citep{phosim}.
\begin{figure}[H]
% Use the relevant command for your figure-insertion program
% to insert the figure file.
% For example, with the graphicx style use
\centerline{\includegraphics[width=2in]{validation_figures/flow.png}}
%
% If no graphics program available, insert a blank space i.e. use
%\picplace{5cm}{2cm} % Give the correct figure height and width in cm
%
\caption{The flow of information through the LSST simulation
  framework. Databases of astrophysical sources are populated with
  models of the cosmological distributions of galaxies, the
  distributions of stars within our Galaxy, and the populations of
  asteroids within our Solar System. Using historical records for the
  weather at Cerro Pachon and the observing cadences required by the
  science drivers for the LSST, sequences of simulated observations
  are generated by the Operations simulator. From these simulated
  pointings, catalogs and images of galaxies can be generated that
  match the expected properties of the LSST system. Comparing the
  catalogs derived by processing the LSST data with those used to
  generate the inputs we enable a full end-to-end test of the LSST
  system.}
\label{fig:flow}       % Give a unique label
\end{figure}


\subsection{Galaxies and Cosmology \label{sec:gal}}

The galaxy model is based on dark matter haloes from the Millennium
simulation \citep{springel05} with an assumed standard $\Lambda$-CDM
cosmology and a semi-analytic baryon model grafted upon the Millennium
results as described in \citet{springel05} and \citet{delucia}. This
semi-analytic model features radiative cooling, star formation, and
the dynamics of black holes, supernovae, and AGNs. It includes
features such as explicitly following dark matter haloes even
after accretion onto larger systems in order to follow the dynamics of
satellite galaxies for an extended period of time as well as 'radio
mode' feedback of AGNs. The model was adjusted to mimic the
luminosity, color, and morphology distributions of low redshift
galaxies \citep{delucia}. LSST cosmological catalogs were generated
from the \citet{delucia} data
by constructing a lightcone, covering redshifts 0$<$z$<$6, from 58
500h$^{-1}$Mpc simulation snapshots. This lightcone covers a 4.5x4.5
degree footprint on the sky and samples halo masses over the range
$2.5\times10^9$ to $10^{12}$ $M_\odot$. 

Dynamically tiling this footprint across the sky enables the
simulation of the full LSST footprint while keeping the underlying
data volume small (but at the expense of introducing periodicity in
the large scale structure).  For all sources, a spectral energy
distribution, is fit to the galaxy colors using Bruzual and Charlot
spectral synthesis models \citep{bruzual}. The
\citet{delucia} catalog includes BVRIK magnitudes and dust values for
the disk and bulge components of each galaxy as well as radii,
redshift, coordinates, stellar age, masses and metallicities. These
are also used in constraining the assignment of SEDs to each disk and bulge.
These fits are undertaken independently for the bulge and disk components and
include inclination dependent reddening. Morphologies are modeled using two
Sersi{\'c} profiles and a single point source (for the AGN) with
bulge-to-disk ratios and disk scale lengths from \citet{delucia}.
 Half-light radii for bulges are estimated using the empirical
absolute-magnitude vs half-light radius relation given by 
\citet{gonzalez09}.  Comparisons between the redshift and
number-magnitude distributions of the simulated catalogs with those
derived from deep imaging and spectroscopic surveys showed that the De
Lucia models under-predict the density of sources at faint magnitudes
and high redshifts. To correct for these effects, sources are
``cloned'' in magnitude and redshift space until their densities
reflect the average observed properties (see \S
\ref{sec:galaxycounts}). 

AGNs are derived using the \citet{bongiorno12} luminosity
function. The B-band absolute magnitudes are converted to bolometric
luminosities using Eqn. 2 in \citet{hopkins07}. Empirical relations
derived from the SDSS enable computation of the colors and stellar
mass of the AGNs host galaxy from its luminosity. These parameters are
used, together with the redshift values from the AGN catalog, to match
each AGN to a galaxy in the galaxy catalog. In general, the AGNs match
to galaxies having higher stellar masses, approximately $10^{9}$ to
$10^{11}$ $M_{\odot}$ which is comparable to recent analysis of host
galaxies done by \citet{xue11}. The AGN SED is taken from the mean AGN
spectrum of \citet{vandenberk}.

\subsection{Galactic Structure \label{sec:stars}}

Stars are represented as point sources and are drawn from the Galfast model of \citep{galfast}.  
Galfast generates stars according to
density laws  derived from fitting SDSS data
to a thick and thin disk along with a halo \citep{juric}. Using an 
input luminosity function measured from SDSS for each class of star 
(main sequence, white dwarf, blue horizontal branch, etc.), Galfast samples stars in space and magnitude 
from a 4-dimensional probability density function
$\rho$(x,y,z,M). After this stage, using Fe/H and kinematics models
from \citet{ivezic08} and \citet{bond09} (also derived from SDSS data), 
each star is assigned a metallicity, proper motion, and parallax.
Spectral energy distributions are fit to the predicted
colors using the models of \citet{kuruczCD} for main sequence
stars and giants, \citet{bergeron95} for white dwarfs,
and a combination of spectral models and SDSS spectra for M, L, and T
dwarfs 
\citep[e.g.][]{cushing05,bochanski07,burrows06,pettersen89,kowalski10}. 
For Galactic reddening, a value of E(B-V) is assigned to each
star using the three-dimensional Galactic model of 
\citet{amores05}. For consistency with extragalactic observations the
dust model in the Milky Way is re-normalized to match the 
\citet{schlegel98} dust maps at a fiducial distance of 100 kpc.  Once the 
extinction and SED are assigned, observed magnitudes are calculated in
the SDSS and LSST photometric systems using fiducial system throughput curves.
Binary stars are included in the luminosity functions from which the
stellar colors are sampled but are assumed to be unresolved and
non-variable (except for a selection of eclipsing binaries described
later).

Stellar  populations included within the current implementation of the model are:
\begin{itemize}
\item Main Sequence: F,G,K,M,L,T
\item White Dwarf: H and He
\item Red Giant Branch
\item Blue Horizontal Branch
\item RR-lyrae
\item Cepheids
\end{itemize}

Approximately 10\% of the stellar sources are variable at a level detectable
by LSST.
Variability is modeled for sources within the base catalogs
by defining a light curve, its amplitude, a period, and a phase. For
queries that contain a time constraints the magnitude of the source is
adjusted based on the properties of the light curve (the current
implementation only allows for monochromatic variations in the
fluxes). Variables modeled range from cataclysmic variables, flaring
M-dwarfs, and micro-lensing events. For transient sources, the period
of the light curve is set to $>10$ years such that the sources will
not repeat within the period of the LSST observations.


\subsection{Solar System \label{sec:ssm}}

The Solar System model is a realization of the \citet{grav11} model.
All major groups of Solar System bodies are represented including:
main belt asteroids, near earth objects, trojans of the major planets,
trans-neptunian objects, and comets. There are approximately 11
million objects in the Solar System catalog with the vast majority (about 9 million) being
main belt asteroids. Populations are complete down to apparent
magnitudes of V=24.5.  Each object is assigned a carbonaceous or stony
composition spectrum derived from extending the reflectance spectra
from \citet{demeo} by linear extrapolation from 4500$\AA$ to 3000$\AA$
and then multiplying by a Kurucz solar spectrum. The choice of a
C or S type spectra for an object is assigned based upon a simple
relation to the size of its orbit that approximately matches SDSS
asteroid observations. Each object's brightness during a specific
observation is calculated from its location, phase, $H_V$ and g
values. $H_V$ is the object's absolute magnitude and corresponds to the
brightness if it were observed at 1 AU from the sun and at zero phase
angle.  The $H_V$ distribution is modeled independently for each
source population (NEO, TNO, main belt, etc.)  as described in \S 3 of
\citet{grav11}.  The g value relates the change in brightness of an
object with the change in phase and is set at 0.15 for all objects
across all bands, which is a typical value for asteroid phase curves.
A more accurate modeling of the asteroid phase curves would require
more realistic rotation and composition models which may be included
in future work.  The location of the Earth at the time of a
particular observation is incorporated through the orbital ephemeris software oorb
(\citet{granvik};{\tt http://code.google.com/p/oorb/}) that calculates a V
band apparent magnitude which is then used with the object's assigned
C or S type SED to derive the corresponding LSST band observations.

\subsection{Query Framework}
The query framework is written in python and takes an object centric view of the 
data.  For each object type (galaxy, main sequence star, strong lens, etc.) a class
is defined that knows how to query for, format, and transform objects of that type.  
This approach provides an extensible framework as different components of the Universal model
are logically distinct and so can be separated into different tables in a database
or in different databases all together.

Many objects are similar in terms of their properties and how they
need to be queried.  This enables a small number of base classes to be
defined that will encompass the requirements of the majority of the
object classes.  Other object classes can then be represented as
simplified subclasses of the base class.  
%Because of this a new object
%class is as simple as overriding the database connection 
%string {\bf  XXX what does this mean}
Extensibility of catalog types is handled by the InstanceCatalog class
which defines how the output from the object classes is formatted into
catalogs.  Again, each catalog  type is a subclass of the base
InstanceCatalog class, so extending to other catalog types simply
requires overriding the default formatter.  This also allows for
custom headers (like those needed for input to phosim) and custom
output (e.g. binary, fits, or Python pickle files).



\section{Validation of the Simulation Framework Requirements}

In the following section we validate the current implementation of the
catalog simulation (catsim) with respect to the LSST simulations
requirements (see \S 4.1 through \S 4.3 in
\citealt{requirements}). Following the structure in the requirements
document we consider the requirements on the properties of the
framework and then the properties of the underlying catalogs used for
the Universe model.  Where the current framework implementations do
not meet the requirements we discuss the impact of this and how we
will resolve this issue through future work.

\subsection{Framework: Requirement 1}

{\it  The simulation framework shall be open source and the input data, configurations,
and software shall be accessible by the LSST project and science
collaborations}

Catsim is released as open source software using the GPLv3 license
\citep{XXX}. The terms of this license are such that the software is
made available for copying, modification, and redistribution
(including derivative works). Adopting GPLv3 ensures that works
derived from the LSST software will also be released under the same
open-source licence. Code developed as part of catsim (including
cofiguration files and example scripts) are stored in the LSST 
git repositories,

\begin{itemize}
\item {\tt https://dev.lsstcorp.org/cgit/LSST/sims/catalogs/generation.git/} -- Code to query
base catalogs.
\item {\tt https://dev.lsstcorp.org/cgit/LSST/sims/catalogs/measures.git/} -- Code to format
and manipulate catalogs produced by the generation packages.  This includes photometric, 
astrometric, and variability calculations.
\item {\tt https://dev.lsstcorp.org/cgit/LSST/sims/throughputs.git/} -- Repository
of system throughputs for the LSST reference design.
\end{itemize}

These repositories are available for anonymous (read-only) checkout or
(with a username and password) for read and write.

The source catalogs for the LSST Universe model are stored in a
Microsoft SQLServer database (SQLServer 2003) that is housed and
maintained at the University of Washington. All data stored within
this database are accessible using the catsim framework and through
standard SQL queries.

\subsection{Framework Requirement 2}

{\it The simulation framework will be documented to a level
  that it can be installed and used by users external to the
  simulation team}

Documentation is available as web pages as well as self documenting
docstrings within the code.  This documentation includes installation
instructions, example applications, a description of the database
schema, and the interfaces and APIs used in catsim. This documentation
is available at the following locations:
\begin{itemize}
\item catalog generation and measures -- {\tt http://www.astro.washington.edu/users/krughoff/documentation/}
\item base catalog schema and description -- {\tt https://dev.lsstcorp.org/trac/wiki/IS\_Catsim\_Database\_Documentation}
\end{itemize}


\subsection{Framework Requirement 3}

{\it  The interfaces between the components of the simulation framework shall be defined 
and documented}

The interfaces between the operations simulator (opsim) and catsim and
between catsim and the photon simulator (phosim) are documented by the
opsim and phosim teams respectively.  These interfaces are described in:
\begin{itemize}
\item OpSim to CatSim -- ({\tt https://lsstcorp.org/sites/default/files/SSTARUserGuide.pdf}; accessed 07/29/2013)
\item CatSim to PhoSim --{\bf XXX where is this}
\end{itemize}

\subsection{Framework Requirement 4} 

{\it The provenance within the data products shall be sufficient to rerun a simulation in a 
deterministic manner}

Provenance is maintain both internally to each component of a
simulator (e.g.\ for catsim) as well as between each of the components
of the simulator framework through a series of mechanisms. For the
Operations Simulator each run of the simulator is given unique
identifier.  The OpSim identifier provides a unique table naming
convention in the OpSim database that is used by the catsim query
framework to select the observing conditions given a pointing and an
OpSim run. 

For catsim, the base catalog data is stored on disk and is
immutable. The base information going into any given run is therefore
the same.  Since the galaxies are tiled across the sky, the identifier
returned by the query framework is a combination of the tile number
and the identifier from the base catalog. This ensures that all
galaxies have a unique id and that that id is the same even if the
observational parameters (e.g. the bore site of the telescope) differ
from query to query.  The place where non-determinism may enter the
process is through the expression of the variability of the
astrophysical sources. Determinism for this process is ensured by
defining the initial time, $T_o$, for the light curve model (i.e.\
essentially specifying the phase of the light curve for a particular
time). A value for $T_o$ is assigned to all variable sources within
the catalog.  Stochastic variability, that uses a random number
generator (e.g.\ the damped random walk of an AGN light curve), have a
common seed for each source.
{\bf XXX is this true}

Special consideration must be paid to keeping ids in the output
catalogs unique.  Since each object database is independent (i.e.\
comprising a separate table or database), the ids will not be unique
when comparing one table to another. The tiling of the galaxy catalog
across the sky introduces a second method for introducing duplicates
in the catalog ids. To overcome these issues we apply a packing scheme
to ensure the original database id is recoverable from the catalog id.
Each object type is given an identifier that is unique across all
tables.  In practice, this is just an integer that is incremented when
a new table is added.  The unique id is then constructed by
bit-shifting the database id (or galtileid in the case of tiled
galaxies) and adding the object type id.  This assumes that there  will
not be more than 1024 object types.  
%The object type identifier is then added
%to the bit shifted id to create a new, unique id.

The input catalogs for the photon simulator includes a seed derived
from the visit id for a particular observation (i.e.\ obshistid from
the opsim database). This seed is used to initialize the photon
simulator and for the photon generation. Since parallelism for the
photon simulator is on the chip level, the order of draws from the
random number generator for each chip is the same.


\subsection{Framework Requirement 5}

{\it The simulation framework shall be capable of running on individual workstations 
and high performance compute clusters}


CatSim runs naturally as a single core process since the problem is
parallelized on a pointing by pointing basis.  To parallelize CatSim
to larger systems many processes are run in a pleasingly parallel mode
with each process running a single pointing.  This approach marries
nicely with schedulers such as the Portable Batch System (e.g.\ {\tt
  http://www.pbsworks.com/}) and Condor \citep{condor}.  The
limitation on the level of parallelism is the database I/O.  The
current hardware (10 core, ??? GB, 40TB raid) has shown to be scalable
up to 70 concurrent processes, resulting in the generation of up to
~200 pointings or visits per day (this represents $\sim$20\% real time
data production).  The throughput of this system can be increase by
duplicating the database server or 
upgrading the hardware.

\section{Validation of the Requirements on the Catalog Simulations}

The simulation catalogs, in conjunction with tools for applying
astrometric and photometric operations and a framework for generating
observed catalogs, provide the requisite tools for conducting a
wide range of analysis activities. For example,
\begin{itemize}
\item the realized positions of solar system objects can be used to
  test moving object detection algorithms.
\item source catalogs can be used to test database and algorithm scaling.
\item inputs generated for phosim using the OpSim observing cadences
  enable large scale image simulations for sensitivity analyses and
  algorithm development.
\item realistic base catalogs enable the injection of specialized objects for specific science
analyses.
\item time domain catalogs of variable objects can be used in testing
  lightcurve recovery and characterization.
\end{itemize}

Each of the examples above requires a set of base catalogs that meet
the project needs.  Where extensions to the types and properties of
sources are required (e.g.\ by the LSST science community), support for
extensions of the query framework is required.

\subsection{Catalogs: Requirement 1}

{\it The LSST catalogs simulations shall contain representations of
  stars, galaxies, quasars, solar system objects, and variable sources
  with properties consistent with the LSST schema}

\subsubsection{Point, Extended and Moving Sources}

We have constructed a database to hold information about the sources
required by the Universal model.  Each of these source types has their
own schema (see sections \ref{sec:gal}, \ref{sec:stars}, and
\ref{sec:ssm} for galaxies, stars, and solar system sources
respectively).

The stellar database contains {\bf XXX} stars, covering the LSST
footprint (Dec $< +30^o$) with tables for main sequence and red giant
branch stars (??? stars), cepheids (), RR Lyrae (), blue horizontal
branch (), and white dwarfs ().  The stellar schema is describe in
{\tt http://ls.st/5jl}. For efficiency, large tables (e.g. the main
sequence stars) are stored in zones each of {\bf XXX} deg wide.  All
tables are indexed with a spherical tree built using a hierarchical
triangular mesh \citep[HTM][]{htm}.

Galaxies are stored in a single table $4.5^o \times 4.5^o$ on a side
with XXX galaxies. This single tile is then replicated across the sky
to provide the spatial coverage needed to simulate the survey (see
Figure \ref{fig:galcoverage}).  Tiling of the galaxies is handled in
the database using a stored procedure. % ({\tt  GalaxySearchSpecColsConstraint2013}).

The Solar System model is the most complicated table in the
database. The most general way to characterize Solar System objects is
through their orbital elements.  Propagating orbits to the time of
observation requires a numerical integration and, given the 11 million
sources in the Solar System model would be computationally prohibitive
for real-time calculation. We, therefore, pre-cache the postions of
asteroids within the database and interpolate their positions based on
the time of the observation.

Ephemerides are calculated for all Solar System sources within the
database for a ten year period. The time between ephemerides is
variable and depends on the asteroid population (i.e.\ it is set by
the velocity of the asteroid and the complexity of its orbital track).
{\bf XXX add the caching frequency from Yusra's email}. In total, {\bf
  XXX} object positions are stored within the Solar System table,
which, using a cubic {\bf XXX} interpolation, returns asteroid
positions with an accuracy of $<XXX$mas (sufficient to meet
requirement {\it Catalogs: Requirements 5}) These cached positions are
indexed using HTM \citep{XXX} to speed spatial lookup.

%, it is very slow with so many objects (11
%million total).  In order to keep query times reasonable the objects
%should be indexed so that it is fast to find objects that could
%possibly be within a particular aperture at a given time.  Ideally the
%index would identify exactly (to the precision required by LSST) the
%location of objects that land in the aperture are located at a given
%time.

For all sources, the generation of magnitudes and color, and the
application of time dependent astrometric corrections (e.g.
precession, parallax, proper motion) are calculated using Python
subclasses of the InstanceCatalog object.

\subsubsection{Variable Sources}
The framework is able to support several types of variability:
periodic, stochastic, and repeating.
The variability models used in the database include:.  
\begin{itemize}
\item M-dwarf flares -- full sky
\item AGN/QSOs -- full sky
\item RRly -- full sky
\item Cepheids -- exemplar individuals
\item Eclipsing binaries -- exemplar individuals
\item Am CVn -- exemplar individuals
\item Micro lensing -- exemplar individuals
\end{itemize}
Each type of variability is described by either a parametric model or
an interpolated lookup table.  To date only mono-chromatic variability
has been implemented (ee Figure \ref{fig:lcs} for example lightcurves).

Variable sources are implemented through the InstanceCatalog API
\citep{XXX}. This API takes the name of the variability model and the
parameters associated with that model (both of which are stored in the
database) and modifies the brightness of a source based on the time of
observation.

\subsection{Catalogs: Requirement 2}

{\it At high Galactic latitudes the average number densities of
  sources shall be within 20\% and 10\% of the observed counts for
  stars and galaxies and galaxies respectively (to the $5\sigma$ point
  source coadded depth of LSST)} 

The motivation for this requirement is two fold.  There is a
functional requirement that the distribution of sources is
representative to test database scaling and I/O sizing models.

Stellar number densities at galactic latitudes closer than to the plane than $|b| < 30^o$ are difficult to model due to the rising density and resulting confusion and deblending issues.
For these reasons, the criterion for stellar number counts at low latitudes are relaxed to $\pm 30\%$. 
For stars where $|b| > 30^o$, the Galfast model has been vetted against the SDSS data.  For stars with $|b| < 30^o$, we quantify the uncertainty in the model by comparing two reference models for Galactic structure.
We compare the realization of the Galfast model to a realization of the \citet{besancon} model.

\subsubsection{Stars}
We take six representative fields at varying galactic latitude at two longitudinal values (one toward the bulge, $b=0$, and one away, $b=180$).  
We compare number 
counts of main sequence stars to the co-added limiting magnitude in i-band 
 from the Galfast model using the composite dust model
of \citet{amores05} normalized to \citet{schlegel98} to the \citet{besancon} model with their dust model.  Figure \ref{fig:scounts_0} shows 
the cumulative number counts as a function of magnitude for the Besan\c{c}on (dashed) and Galfast (solid) models 
for six values of galactic longitude toward the galactic bulge.  We are interested in the
fractional cumulative contribution.  In Figure \ref{fig:sratio_0} we plot the ratio of Besan\c{c}on counts to Galfast counts for the six test fields shown in Figure \ref{fig:sratio_0}.
The dashed lines are the $\pm30\%$ limits for the low latitude sizing model constraints and the dash-dot lines are the $\pm18\%$ limits for the high latitude limits.
Since the number counts are dominated by the faint end, it's most important where the lines end up at faint magnitudes for sizing considerations 
(the co-added depth for the i-band is 26.8).  
Nominal single epoch depth for i-band is 24.0.  

In the case directed toward the galactic bulge, all fields meet the requirements at the co-added depth except
for $b=-10$.  At the nominal single epoch depth, the $b=-10$ case fails and the $b=-30$ case misses the $\pm18\%$
requirement, but makes the $\pm30\%$ requirement.

For the high latitude fields away from the galactic bulge, the deviation from the Besan\c{c}on model is within 
the stated requirement.  The $b=-10$ case, the number counts from Galfast significantly under predict relative to 
the Besan\c{c}on model.  The $b=-30$ case shows similar, though not as drastic, under prediction and misses the 
requirement for all but the faintest magnitudes and even then only meats the $\pm30\%$ requirement.

%We repeated this analysis using the GALFAST dust model and the results do not change significantly
\begin{figure}[H]
\centering
\includegraphics[width=5in]{validation_figures/cumulative_stars_0_besancon_dust.png}
\caption{Cumulative counts of stars from the Besan\c{c}on (dashed) and Galfast (solid) models for 5 representative fields toward the Galactic bulge (l=0$^o$) \label{fig:scounts_0}}
\end{figure}
\begin{figure}[H]
\centering
\includegraphics[width=5in]{validation_figures/cumulative_ratio_stars_0_besancon_dust.png}
\caption{Cumulative ratio of counts of stars from the Besan\c{c}on and Galfast models for 5 representative fields toward the Galactic bulge (l=0$^o$) \label{fig:sratio_0}}
\end{figure}
\begin{figure}[H]
\centering
\includegraphics[width=5in]{validation_figures/cumulative_ratio_stars_180_besancon_dust.png}
\caption{Cumulative ratio of counts of stars from the Besan\c{c}on and Galfast models for 5 representative fields away from the Galactic bulge (l=180$^o$) \label{fig:sratio_180}}
\end{figure}

\subsubsection{Galaxies \label{sec:galaxycounts}}
We compare the galaxy counts to those provided by Metcalfe et al. (see {\tt http://star-www.dur.ac.uk/~nm/pubhtml/counts/counts.html}; 
accessed 07/29/2013).  We have taken their compilations from:
{\tt http://star-www.dur.ac.uk/~nm/pubhtml/counts/idata.txt} accessed on 06/01/2013.  
We include those data points with error bars.  
Using these counts
we noticed that the numbers of galaxies were under predicting at faint magnitudes.  

The result of the number counts as a function of magnitude (after the cloning correction see \S \ref{sec:gal}) are shown in Figure \ref{fig:gcounts}.  
We have chosen the i-band data for comparison to 
minimize the effects of dust extinction which are somewhat uncertain in the Durham compilations.  A single transform from Kron-Cousins to AB of I$_{kc}$ = i$_{AB}$ - 0.5 was applied to
all compilation data.  For comparison with requirements on the sizing model stated in the requirements document, we also plot the cumulative ratio of the best fit polynomial
to the Durham data to the counts from the base catalog.  The requirement is $\pm18\%$ to the coadded i-band depth of 26.8.  We see that the base catalog
under-predicts at the faintest magnitudes, but meets this requirement (Figure \ref{fig:gratio}).

\begin{figure}[H]
\centering
\includegraphics[width=5in]{validation_figures/Ngals-i.png}
\caption{Durham counts (symbols) compared to the counts from the base catalog \label{fig:gcounts}}
\end{figure}
\begin{figure}[H]
\centering
\includegraphics[width=5in]{validation_figures/CumulativeFraction_i.png}
\caption{Durham counts divided by the counts from the base catalog.  Error bars are from adding the error bars from the data points in quadrature. \label{fig:gratio}}
\end{figure}

\subsubsection{Ramifications of missing requirements}
At low galactic latitudes the models for the stars are uncertain at a level greater than the requirements (i.e the discrepancy between Galfast and Besan\c{c}on is > {\bf XX}\%).  This has the potential to impact the sizing and compute models for the LSST.

The validity of the Besan\c{c}on as ground truth in the plane of the galaxy is not obvious.  In fact, SDSS and Galfast star counts agree surprisingly well at low galactic latitudes (private communication with M. Juri\'{c}). 
However, the completeness of SDSS deep in the plane has also not been quantified.  Certainly in the densest regions
of the plane there is significant loss of stars due to blending.  Further, since the Besan\c{c}on models are drawn
from distributions with binary corrections applied and the Galfast model does not produce unresolved binary systems
as multiple objects in the catalogs, the Galfast models will under-predict relative to the Besan\c{c}on by 
approximately the unresolved binary fraction.  These questions will be resolved through an ongoing comparison to observations
from other surveys (e.g. Pan-STARRS).

\subsection{Catalogs:  Requirement 3: Size, ellipticity, and redshift
  distributions of galaxies shall be representative of those observed
  by extant space and ground-based telescopes and, for a fiducial
  image quality of 0.72 arcsec, deviations from the observed
  distributions shall contribute $<20$\% of the observed effective
  density of galaxies, n$_{eff}$, used in the weak lensing samples
  (assuming a fiducial value of n$_{eff} = 28$ galaxies per
  arcmin$^2$)}
\subsubsection{Motivation}
The measurement of weak lensing is dependent 
on $n_{eff}$ which is the effective density of galaxies on the sky that can be used to measure weak lensing.  
The value of $n_{eff}$ depends on the inherent shape noise, the signal-to-noise distribution and the size distribution of galaxies relative to the PSF.



\subsubsection{Measuring $n_{eff}$}
We use the framework described in \citet{chang} to calculate $n_{eff}$ for the distributions in the 
base galaxy catalog.  In summary, we use Equation 9 to calculate $n_{eff}$:
\begin{equation}
n_{eff} = \frac{1}{\Omega}\sum^N_i\frac{\sigma^2_{SN}}{\sigma^2_{SN}+\sigma^2_{m,i}}
\end{equation}
where $\sigma_{SN}$ is the intrinsic shape noise and $\sigma_{m,i}$ is the shape measurement noise for the i$^{th}$ galaxy.

The shape noise is derived from the ellipticity distribution.  For the distribution used in \citet{chang} $\sigma_{SN} = 0.26$.
The measurement noise can be approximated using Equation 13:
\begin{equation}
\sigma_m(\nu,R) = \frac{a}{\nu}\left[1+\left(\frac{b}{R}\right)^c\right]
\end{equation}
Where $\nu$ is the signal-to-noise ratio of the source, $R=\frac{r_{gal}^2}{r_{PSF}^2}$ is the size of the galaxy relative to
the point spread function.  We adopt values from \citet{chang} of (a,b,c) = (1.58,5.03,0.39), and assume a fiducial
PSF size of 0.7 arcsec. We use an estimate for the limiting magnitude of the co-added images
of 26.7.  This takes into account the fact that the measurements are on extended sources as well as the fact that the SRD value 
of 27.5 in r is for dark sky observations at zenith.   The galaxy sample is the ``Gold Sample'' of galaxies with $i < 25.3$.
For each galaxy, $r_{gal}$ can be calculate from the flux ratio of the bulge and disk components as well as the effective half light 
radii of each component (see Appendix B in \citet{chang} for a derivation of this calculation).

Using this framework, we test the sensitivity of the measured $n_{eff}$ on the ellipticity and size distributions.  For all calculations that follow we 
use the k=1 criterion of \citet{chang} meaning that galaxies with $\sigma_m < \sigma_{SN}$ are culled from the sample.

The shape noise distribution was well measured by the COSMOS project \citep{cosmos} and, to first order, is just the variance in the measured 
ellipticity distribution.  The $n_{eff}$ depends strongly on the apparent magnitude because of the steepness of the galaxy number
counts, but because of the dependence on signal to noise it has
has a very sharp cutoff at around i=24.5 (see Figure \ref{fig:neffvm}).  Galaxy redshift distributions must
agree with observations to this limiting magnitude to assure accurate signal to noise distributions.  Finally, the size distribution
of the combined two component galaxy model must match measured distributions closely in order to reproduce predicted
values of $n_{eff}$ from measured distributions.  We measure the distribution from the base catalogs and show that it is well
within the envelope necessary to reproduce realistic values of $n_{eff}$.

The ellipticity distribution is measured from the base catalog by sampling the appropriate composite Sersi{\'c} (bulge and disk together) using the
ellipsoid parameters and flux ratio for each galaxy.  The moments are measured and ellipticities are calculated using the definitions in \citet{chang}.
\begin{equation}
\epsilon_1 = \frac{I_{11}-I_{22}}{I_{11}+I_{22}+2\sqrt{I_{11}I_{22} - I^{2}_{12}}}\\
\epsilon_2 = \frac{2I_{12}}{I_{11}+I_{22}+2\sqrt{I_{11}I_{22} - I^{2}_{12}}}
\end{equation}
We measure the same value of shape noise as reported by \citet{chang}, $\sigma_{SN} = 0.26$.  The shape appears to be slightly different than
that of the COSMOS sample, but we have not investigated the effect of distribution shape on the measured value of $n_{eff}$ (see Figure \ref{fig:ellip1} 
and Figure \ref{fig:ellip2}).  
Slight variations in
ellipticity distribution shape will not impact the measured value of $n_{eff}$ enough for the catalog to miss the requirement.  Figures \ref{fig:ellip_errbig}
and \ref{fig:ellip_errsmall} show the measured ellipticity distribution from the base catalog scaled such that the $n_{eff}$ requirement would be missed on the high side 
and on the low side respectively.
\begin{figure}[H]
\centering
\includegraphics[width=5in]{validation_figures/e1_hist.png}
\caption{Histogram of e1 values for the base catalog and COSMOS sample.  These distributions are for objects with $20.0 < i < 24.5$ to match
the cut made in \citet{chang}.\label{fig:ellip1}}
\end{figure}
\begin{figure}[H]
\centering
\includegraphics[width=5in]{validation_figures/e2_hist.png}
\caption{Same as Figure \ref{fig:ellip1} but for e2.\label{fig:ellip2}}
\end{figure}

\begin{figure}[H]
\centering
  \includegraphics[width=5in]{validation_figures/e1_hist_s_354.png}
\caption{The base catalog distribution with the width scaled such that the requirement is just missed by predicting an $n_{eff}$ value that is too large.\label{fig:ellip_errbig}}
\end{figure}
\begin{figure}[H]
\centering
  \includegraphics[width=5in]{validation_figures/e1_hist_s_17.png}
\caption{The base catalog distribution with the width scaled such that the requirement is just missed by predicting an $n_{eff}$ value that is too small.\label{fig:ellip_errbig}}
\end{figure}

For completeness, we show the redshift distributions for 1 magnitude bins from $i=18$ to $i=24$.  See Figure \ref{fig:nofz18_24}.  As shown in \ref{fig:neffvm}
galaxies with $i > 25$ do not contribute significantly to $n_{eff}$. 
\begin{figure}[H]
\centering
\includegraphics[width=5in]{validation_figures/Nofz_18_24.png}
\caption{The blue histogram in each panel is the measured N(z) for 1 magnitude bins from $18<i<24$.  Overplotted is the empirical distribution from \citet{coil04} normalized to the area of the measured distribution.  The dashed and dotted red lines show what happens if the governing parameter for the \citet{coil04} distribution, $z_o$, is over
or under predicted by $3\sigma$ respectively.\label{fig:nofz18_24}}
\end{figure}
%\begin{figure}
%\centering
%\includegraphics[width=5in]{validation_figures/Nofz_CumulativeFraction_18_24.png}
%\caption{N(z) for 18 to 24 with distribution from \cite{coil}\label{fig:nofz18_24_ratio}}
%\end{figure}
%\begin{figure}
%\centering
%\includegraphics[width=5in]{validation_figures/Nofz_coil_22_28.png}
%\caption{N(z) for 22 to 28 with distribution from \cite{coil}\label{fig:nofz22_28}}
%\end{figure}
%\begin{figure}
%\centering
%\includegraphics[width=5in]{validation_figures/Nofz_CumulativeFraction_22_28.png}
%\caption{N(z) for 22 to 28 with distribution from \cite{coil}\label{fig:nofz22_28_ratio}}
%\end{figure}
\begin{figure}[H]
\centering
\includegraphics[width=5in]{validation_figures/neff_m_ir.png}
\caption{This shows $n_{eff}$ as a function of i-band limiting magnitude.  Even though the the density of galaxies is going up steeply, the 
size of the galaxies relative to the PSF is falling off quickly enough that galaxies fainter the $i=25$ don't contribute much to $n_{eff}$.\label{fig:neffvm}}
\end{figure}

{\bf XXX read to here}

\subsubsection{Galaxy radius measurements}
There are many definitions of galaxy size: half light radius, first moment radius, second moment radius, Petrosian radius, Kron radius, etc.
Our comparison data set is that from the COSMO ACS catalog \citep{cosmos} because of the size ($~2deg^2$), depth ($~26$ in i), and
quality (space based).  

The COSMOS catalog reports the second moments and the 
half light radius, so either the second moment radius or half light radius could be used for comparison.  One concern is that the measurements
on the base catalog are effectively infinite signal to noise, whereas the measurements from COSMOS are not.  
We consider the impact of the S/N on the derived quantities in order to define the appropriate measure to use for comparison.
To measure the sensitivity of the half light radius and second moment radius to the effects of S/N we 
sample from truncated Sersi{\'c}
profiles for each galaxy.  Measurement on a truncated profile is an estimate of the same measurement made on a noisy profile that drops below
the noise at the truncation radius as the noise should contribute to the moments symmetrically.
We truncate the profiles at several multiples of the half light radius, $R_{hl}$: $1.33R_{hl}, 1.78R_{hl}, 3.16R_{hl}, 10R_{hl}, $and$100R_{hl}$. 
Then we measure the half light radius and second moment radius distributions.

Figures \ref{fig:hl_hist} and \ref{fig:mom_hist} show the qualitative effect of making measurements on progressively lower signal-to-noise
profiles.  As the signal-to-noise goes down we see the peak of the distribution and the width of the distribution decreases in both the 
half light radius and second moment radius distributions.  Figure \ref{fig:mom_hl_line} shows this trend.  Going from $100R_{hl}$ to $10R_{hl}$
the mean of the second moment radius distribution decreases by $0.14\%$ and is half the high signal-to-noise value at $1.78R_{hl}$.  By comparison,
the mean of the half light radius distribution decreases by less than $1\%$ going from $100R_{hl}$ to $10R_{hl}$ and is $22\%$ of the high 
signal-to-noise distribution.  The width of the distributions exhibit similar behavior with the second moment radius distribution
impacted much more heavily by reduced signal-to-noise than does the half light radius distribution.  Because of the stability
of the half light radius distribution in the presence of noise, we choose to compare the half light radius as the size quantity
to compare to observed data sets.
\begin{figure}[H]
\centering
\includegraphics[width=5in]{validation_figures/half_light_hist.png}
\caption{The measured half light distribution on truncated Sersi{\'c} models.  The truncation for each distribution is listed in the legend along with the
mean and standard deviation at that value of the truncation.  Truncating the Sersi{\'c} at smaller radii approximates lower signal-to-noise measurements. \label{fig:hl_hist}}
\end{figure}
\begin{figure}[H]
\centering
\includegraphics[width=5in]{validation_figures/Second_moment_hist.png}
\caption{Same as Figure \ref{fig:hl_hist} but for the second moment radius distribution.\label{fig:mom_hist}}
\end{figure}
\begin{figure}[H]
\centering
\includegraphics[width=5in]{validation_figures/sec_mom_half_light_mean_sigma.png}
\caption{The values from the legends in Figures \ref{fig:hl_hist} and \ref{fig:mom_hist} plotted as lines so the trends in mean and standard deviation as a function 
of signal-to-noise are more obvious.  Signal-to-noise is decreasing to the left on this plot.\label{fig:mom_hl_line}}
\end{figure}

To test the sensitivity of $n_{eff}$ on the input half light radius distribution, we model the the half light distribution as a log normal distribution.  Figure
\ref{fig:ln_fit} shows log-normal fits to the base catalog and COSMOS data half light radius distributions.  The histograms are the data and the black lines are the
log-normal fits to the histograms normalized to have the same area.  The half light radius distribution for the base catalog is dominated by the disk components 
since there  are many more disk only objects than
bulge only objects and the disk components have large half light radius values (relative to the bulge components).  To simulate varying half light radius distributions
we take the bulge distribution from the base catalog and then
sample from a log-normal distribution with varying shape parameters.
\begin{figure}[H]
\centering
\includegraphics[width=5in]{validation_figures/ln_fit.png}
\caption{{\bf XXX This figure needs to be remade, it was done with only objects 20 - 24.5.}\label{fig:ln_fit}}
\end{figure}

Figure \ref{fig:size_sens} shows the sensitivity of $n_{eff}$ to the size distribution.  The x axis is related to the width of the 
size distribution and the y axis is related to the location of the peak of the size distribution. Over-plotted are points corresponding
to the best fit log normal distributions to the COSMOS data set (black circle) and the base catalog data set (black square).  The figure has been
normalized to the COSMOS distribution as truth.  This shows that the difference between the COSMOS distribution and
the base catalog distribution makes a difference less than 2 in the value $n_{eff}$ predicted by the base catalog.  This is well within the $\pm6$ in $n_{eff}$ 
stated by the requirements.
\begin{figure}[H]
\centering
\includegraphics[width=5in]{validation_figures/size_sensitivity.png}
\caption{Sensitivity of the predicted $n_{eff}$ to the shape of the half light radius distribution.  
The surface is normalized to the value of $n_{eff}$ as measured using the half light radius distribution from COSMOS.  The black circle
is the location of the COSMOS data in this plane which by construction falls on the zero contour.  The black square is the prediction of
the measured $n_{eff}$ using the half light radius distribution from the base catalog.\label{fig:size_sens}}
\end{figure}


\subsection{Catalogs: Requirement 4: For the photometric calibration simulations
the distribution of stellar colors shall encompass the colors of white dwarfs through red giant branch stars.
The median color distributions of stars must trace the observed color locus for these stars to within 0.02 magnitudes
of the principal color (s,w,x,y) to the designed 5$\sigma$ single epoch limiting magnitude in the r-band.
{\bf update this for the most recent statement in the requirements document}}
In order to evaluate the sensitivity of the photometric calibration to the distributions of stars within a focal plane, the Galactic model
must have realistic color distributions.
This goes beyond simply spanning the appropriate color ranges.  The main sequence stellar locus must agree with the location of the
stellar locus as measured by other projects.  In Figure \ref{fig:starcolorspan} we verify that we meet the requirement that the stellar
colors span the ranges given in the requirements document is met.  For each color ($u-g$, $g-r$, $r-i$, $i-z$, and $z-y$) we plot the normalized histogram
for the main sequence and red giant branch (RGB) stars (forward hatching) and the white dwarf (backward hatching).  Together the two distributions span the required
range as shown by the dashed vertical lines in each panel.  
\begin{figure}[H]
\centering
\includegraphics[width=5in]{validation_figures/star_lsst_color_hist.png}
\caption{Normalized counts of main sequence, red giant branch and white dwarf stars as a function of color.  Heavy dashed lines show the requirements given in the requirements document.\label{fig:starcolorspan}}
\end{figure}

To verify the veracity of the main sequence stellar locus, we use the principal colors of the stellar locus defined by \citet{helmi02} and 
fit for in the SDSS photometric system by \citet{ivezic04}.
We use stars selected from the same fields as used in the number counts analysis for all fields with $b<-30$.
In order to avoid complications associated with the difference between the LSST and SDSS photometric systems, we calculate the un-extincted 
magnitudes in the SDSS bandpasses using the best fit spectrum for each star.  We then calculate
the principal colors for each star using the relations in \citet{ivezic04}, but removing the r-band dependence in $P\prime_{2}$.  The correction
in \citet{ivezic04} was used to correct for an empirical effect that is not present in photometry calculated on idealized spectra with idealized throughput
curves. Figure
\ref{fig:principalcolorshist} shows that the principal colors as calculated from the base catalog are in good agreement with
the location of the stellar locus in the SDSS (zero color).  The base catalog easily meets the requirement of $\pm0.02mag$ deviation
from the stellar locus to single epoch depth in all 4 principal colors: ${\Delta}s=-0.003, {\Delta}w=0.001, {\Delta}x=-0.005, {\Delta}y=-0.013$.  
The increase in scatter in the s band when compared to other colors arises from a metallicity dependence in the color.  Figure \ref{fig:sfeh} shows this 
dependence which is consistent with SDSS observations.
\begin{figure}[H]
\centering
\includegraphics[width=5in]{validation_figures/principal_colors_hist.png}
\caption{Histograms of the principal colors of stars in the base catalog compared to SDSS principal colors (zero color in these plots) to the stretch single epoch 
depth of $r < 24.8$. The mean and standard deviation for each 
histogram are given in the legend in the upper right.\label{fig:principalcolorshist}}
\end{figure}
\begin{figure}[H]
\centering
\includegraphics[width=5in]{validation_figures/s_met.png}
\caption{The principal color s as a function of metallicity.\label{fig:sfeh}}
\end{figure}


Figure \ref{fig:principalcolorshist} shows that the requirement of mean deviation from the color locus defined by the four principal colors by less than 
0.02 magnitudes is met in all four bands.

\subsection{Catalogs Requirement 5: All models for the astrometric transforms applied to the catalogs (including interpolation functions) 
shall have an accuracy better than 1.6 mas}
All operations on positions are undertaken using double precision accuracy. Angular operations are converted to Cartesian coordinates to provide accuracy at the celestial poles and to minimize the complications of the coordinate wraparound at the meridian. Astrometric accuracy of the CatSim simulation framework is determined from the performance of the Starlink positional astronomy library (SLALIB; \citet{wallace}). This software library enables astrometric transformations and operations with
accuracies on the order of a milliarcsecond or better.  The principal SLALIB routines used by the CatSim framework are those for precession, nutation, stellar aberration, rotation of the Earth, diurnal aberration, and refraction.  As described in \citet{wallace} the definition of ICRS coordinate system used in SLALIB and its transform to an FK5 (Fifth Fundamental Catalog - The Basic Fundamental Stars) system has an rms accuracy that is sub-milliarcsecond. Precession and nutation are based on the
model of \citet{SF2001} and have an rms uncertainty of $<$1 milliarcsecond (with the uncertainty growing at approximately 0.3 milliarcseconds per 1000 years). Aberration and light deflection corrections are undertaken in an iterative manner with uncertainties of $<$1 milliarcsecond and the Earth position and velocity corrections are based on the methods of \citep{stumpff} with a maximum error of 0.3 milliarcseconds. All of these components, added in quadrature, meet the requirements described
in the simulation requirements document \citet{requirements}.

The largest of the uncertainties in any of the astrometric transforms  re those that arise due to the color dependence of refraction. For a nominal set of atmospheric conditions (i.e.\ temperature, pressure, humidity, and lapse rate) the rms uncertainty ranges from 0.3 to 1 $\mu$m is 2.5 milliarcseconds. While refraction can be applied for science catalogs, for the catalogs generated as input to the LSST photon simulator refraction is not applied to the positions (i.e.\ it is generated internally to the photon
simulator).
\subsection{Catalogs: Requirement 6: The system shall be capable of incorporating new astrophysical catalogs without requiring
a redesign of the class-schema framework}
The ability to create new catalogs is available within the framework through the use of of user-designed subclasses and class mixins. A new type of catalog can be generated as a new class using the InstanceCatalog base class and other user-designed catalog classes and class mixins. The columns required in the new catalog are defined as class attributes. The data for these columns is gathered from the database directly or using methods defined in the class
for the catalog itself, in class mixins designed by the user, or in the base classes. These data gathering methods can take the form of a {\tt get\_[column\_name]} method or in the form of compound columns with their own {\tt get\_} methods. The required database column is then referred to in the {\tt column\_by\_name()} method which returns the column values either using the appropriate {\tt get\_} method or from the result of a database query. The appropriate method to use is determined by
introspection on the InstanceCatalog object. For example, the CustomCatalog below exercises these aspects of this framework to write a new catalog to file:

\begin{verbatim}
class BasicCatalog(InstanceCatalog):
    """Simple catalog with columns directly from the database"""
    catalog_type = 'basic_catalog'
    refIdCol = 'id'
    column_outputs = ['id', 'ra_J2000', 'dec_J2000']
    # transformations specify conversions when moving from the database
    # to the catalog.  In this case, we take RA/DEC in radians and convert
    # to degrees.
    transformations = {"ra_J2000":np.degrees,
                       "dec_J2000":np.degrees}

class AstrometryMixin(object):
    @compound('ra_corrected', 'dec_corrected')
    def get_points_corrected(self):
        ra_J2000 = self.column_by_name('ra_J2000')
        dec_J2000 = self.column_by_name('dec_J2000')
    # ... do the conversions: these are just standins
    ra_corrected = ra_J2000 + 0.001
    dec_corrected = dec_J2000 - 0.001
    return ra_corrected, dec_corrected

class CustomCatalog(BasicCatalog, AstrometryMixin):
    catalog_type = 'custom_catalog'
    refIdCol = 'id'
    column_outputs = ['id', 'redshift', 'points_corrected']
    transformations = {"ra_corrected":np.degrees,
                       "dec_corrected":np.degrees}

# Now to create a catalog, we connect to a database and call write_catalog
db = GalaxyObj()
catalog = CustomCatalog(db)
catalog.write_catalog("out.txt")

# out.txt has the following columns:
# id redshift ra_corrected dec_corrected
\end{verbatim}

The InstanceCatalog metaclass verifies that all new columns desired in the user-defined catalog have associated means of gathering the required data from the database. If no method exists nor is there an associated database entry for the column directly then an error is raised. Furthermore, the error is raised before data begins transfer from the database because the instantiation of the class initiates a dry run of the table output. After the dry run the necessary database columns are verified
against the actual columns in the database and if there is a discrepancy the error is raised. As a result, the user is protected from errors after the data is pulled from the database and since all the necessary columns are determined during the dry run only one single query of the database is required to pull all necessary data.
\section{Future work}
We expect the requirements of the project to continue to evolve.  Informed by interations with the Systems Engineering team and the Data Management team we have 
identified several areas in which future development will likely take place.  Below we itemize these likely areas of future work.
\begin{itemize}
\item Implement bright stars -- For guiding and wavefront sensing analysis
\item Implementation of extended and morphological images -- Multifit, supernova analysis, galaxy models
\item Reduction of SED data size using PCA -- Parallelization of catalogs generation
\item Add errors to catalogs -- Calibration simulations
\item Add SNe to the catalogs to the framework -- Supernova sensitivity analysis
\item Increase size (area) of galaxy catalogs including cosmological signatures -- Large scale structure
\item Add weak lensing to the catalogs -- Weak lensing
\item Add extended stellar sources, e.g. clusters -- Deblending
\item General variability model -- Alert producton and light curve characterization
\item Further validation of the Milky Way model at low galactic latitudes using Pan-STARRS
\item SNe
\item Proper motion induced by binaries
\item Pan-chromatic variability with spectro-temporal surfaces
\end{itemize}
\bibliographystyle{plainnat}
\bibliography{validation}
\end{document}
